\documentclass[]{article}
\usepackage{amsmath}
\usepackage{layout}

\title{measuring the effectiveness of a proposed algorithm for converting image files into sounds on hearing participants \\(first rough draft)}
\begin{document}
\maketitle
\section{abstract}
here we present an algorithm to convert image files into audible tones. specifically for converting files stored in the .bmp file format into .wav sound files.
along with a behavioral analysis into its effectiveness in aiding participants with associating specific sounds with colors
\section{algorithm}
Generally, uncompressed Bitmap files (Kirkby, bmp format) are organized in such a manner where 1 byte (able to represent numbers up to 255) 
is used to store color values (asuming 24 bit color depth), specifically denoting the red, green, and blue light spectrums (often denoted as r,g, and b). 
\\
\begin{math}
    C = \begin{pmatrix}
            r & g & b
        \end{pmatrix}
\end{math}
\\

(describe wave format here find reference or else elma says no cock for a week)

A constinant \begin{math} k \end{math} will be multiplied with each corisponding component of the color vector. 
\begin{math} k \end{math} in itself, is a scaling factor that mainly exists to denote individual color
values as distinct frequencies. 
\\\\
Next, \begin{math}  \overrightarrow{H} \end{math} , which is a vector of constinants that represent different base frequencies for the specific color spectrums
\\\\
\begin{math} 
    \overrightarrow{H} = \begin{pmatrix}
            H_1 & H_2 & H_3
        \end{pmatrix}
\end{math}\\
\begin{math} h_1 \end{math} through \begin{math}h_3 \end{math} define respective baselines for audible tones  (in hz) to determine how the color will sound after being
multiplied with a scaling factor
Which brings us to the following equation, represented by function \overrightarrow{R}()
\\
\begin{math}
    \overrightarrow{R}(\overrightarrow{C},\overrightarrow{H},k) = \overrightarrow{C}k + \overrightarrow{H}
\end{math}
\\\\
\section{calibration?}
For this study we will assign \begin{math} \overrightarrow{H} \end{math} and k with the following values
\\\\
\begin{math}
    k = 3 \\
    \overrightarrow{H} = \begin{pmatrix} 1200,2200,3200 \end{pmatrix}\\
\end{math}
\\\\
Based on the results of controll participants in the study “Tone deafness: a new disconnection syndrome?.”(Loui, Psyche et al),
non tonedeaf persons showed a minumum threshold of 3hz when it came to identifying unique tones. Hence 3 was identified to be an appropriate scaling factor 
\\\\
The values for \begin{math} \overrightarrow{H} \end{math} were determined from the results of the study "Exploring Pitch Perception Thresholds in Very Short Tones: 
Minimum Tone Duration for Perception of Pitch in 2-Tone and 3-Tone Sequences." (Burton, Russell,104), that suggests that tones played at higher frequencies
are much more decernable than tones played at lower frequencies(in a range from 250hz to 1050hz). Allthough the research paper did not go into specific detail on
how well participants were able to discern frequencies above 1050hz, it could be assumed that ranges higher than 1050hz still retain better decernability over tones
played at lower frequencies.

\begin{thebibliography}{9}
Kirkby , David. “Bmp Format.” Bitmap Format., University College London, users.cs.fiu.edu/~czhang/teaching/cop4225/project_files/bitmap_format.htm. Accessed 3 June 2024.
Humes, Larry E. “Hearing Thresholds for Unscreened U.S. Adults: Data From the National Health and Nutrition Examination Survey, 2011–2012, 2015–2016, and 2017–2020.” Trends in Hearing, vol. 27, Mar. 2023, p. 23312165231162727. PubMed Central, https://doi.org/10.1177/23312165231162727.
Burton, Russell. "Exploring Pitch Perception Thresholds in Very Short Tones: Minimum Tone Duration for Perception of Pitch in 2-Tone and 3-Tone Sequences." 2021. Thesis. digital.library.adelaide.edu.au, https://digital.library.adelaide.edu.au/dspace/handle/2440/134561
Loui, Psyche et al. “Tone deafness: a new disconnection syndrome?.” The Journal of neuroscience : the official journal of the Society for Neuroscience vol. 29,33 (2009): 10215-20. doi:10.1523/JNEUROSCI.1701-09.2009
\end{thebibliography}

\end{document}